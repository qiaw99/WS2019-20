\documentclass{article}
\title{ALP3 U14}
\author{Übungsgruppe: Qianli Wang und Nazar Sopiha}
\date{\today}
\usepackage{amsmath}
\usepackage{amssymb}
\usepackage{listings}
\usepackage{color}
\usepackage{graphicx}
\usepackage{booktabs} 
\definecolor{dkgreen}{rgb}{0,0.6,0}
\definecolor{gray}{rgb}{0.5,0.5,0.5}
\definecolor{mauve}{rgb}{0.58,0,0.82}

\lstset{ %
	language=Octave,                % the language of the code
	basicstyle=\footnotesize,           % the size of the fonts that are used for the code
	numbers=left,                   % where to put the line-numbers
	numberstyle=\tiny\color{gray},  % the style that is used for the line-numbers
	stepnumber=2,                   % the step between two line-numbers. If it's 1, each line 
	% will be numbered
	numbersep=5pt,                  % how far the line-numbers are from the code
	backgroundcolor=\color{white},      % choose the background color. You must add \usepackage{color}
	showspaces=false,               % show spaces adding particular underscores
	showstringspaces=false,         % underline spaces within strings
	showtabs=false,                 % show tabs within strings adding particular underscores
	frame=single,                   % adds a frame around the code
	rulecolor=\color{black},        % if not set, the frame-color may be changed on line-breaks within not-black text (e.g. commens (green here))
	tabsize=2,                      % sets default tabsize to 2 spaces
	captionpos=b,                   % sets the caption-position to bottom
	breaklines=true,                % sets automatic line breaking
	breakatwhitespace=false,        % sets if automatic breaks should only happen at whitespace
	title=\lstname,                   % show the filename of files included with \lstinputlisting;
	% also try caption instead of title
	keywordstyle=\color{blue},          % keyword style
	commentstyle=\color{dkgreen},       % comment style
	stringstyle=\color{mauve},         % string literal style
	escapeinside={\%*}{*)},            % if you want to add LaTeX within your code
	morekeywords={*,...}               % if you want to add more keywords to the set
}

\begin{document}
\maketitle

\section{Aufgabe 92: Adjazenzlisten}
\textbf{Inspirieren durch Radixsort:}\\
Dieser Algorithmus ist schon im Vergleich zum Radixsort vereinfacht, weil in allen Adjazenzlisten jede Kante gleiches i hat.\\
Wir können dann folgendes machen:\\
1. Verteile jedes j in jeder Adjazenzliste in 10 Fächern nach der letzten Ziffer  \\
2. Füge Fach 0, 1, ..., 9 zu einer Liste zusammen.\\
3. Iterieren m mal, wobei m die Anzahl von Bits von Maximum in der Adjazenzliste ist.

\section{Aufgabe 93: Optimaler binärer Suchbaum}
\textbf{Vom Algorthmus erzeugte Tabelle:}\\
\includegraphics[scale=0.4]{test.png}\\
Die Zwischenwerte im Algorithmus, also ${S_i}_j$ in der Tabelle bedeuten das Kosten des optimalen Suchbaums für Schlüssel im Intervall $x_i < x < x_j$. \\
Die mittlere Suchzeit: $\frac{37}{20}$\\
\\
\\
\\
\\
\\
\\
\\
\textbf{Optimaler binärer Suchbaum:}\\
\includegraphics[scale=0.6]{test1.png}\\




\section{Aufgabe 97: Zahlen mit Münzen}
\textbf{Wir nehmen an, dass man möglich weinige Münze bekommen sollte.} \\ 
\textbf{Teilprobleme:} $A_k$ bedeutet die Liste von Münzen mit dem zu zahlenden Betrag k. \\
\textbf{Randbedingung:} $A_{M_i} = [M_i]$\\
\begin{lstlisting}
initialize(betrag, M):	
	memo = [[NIL] * betrag]
	for each m in M:
		memo[m] = [m]
	endfor
	
changeMoney(betrag):
	res = {}
	if(memo[betrag] != NIL):	//Memoization
		extend memo[betrag]	 into res	// this is a list
		return 
	endif
	
	for i = (m.length - 1) to 0 do:
		if(M[i] <= betrag):
			append M[i] into res	// this is a number
			return changeMoney(betrag - M[i])
		endif
	endfor
	
\end{lstlisting}
Python Code sehen Sie unter: $https://github.com/qiaw99/WS2019-20/blob/master/DataStructure/U14/U14_97.py$
\end{document}
