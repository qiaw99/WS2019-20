\documentclass{article}
\title{ALP3 U13}
\author{Übungsgruppe: Qianli Wang und Nazar Sopiha}
\date{\today}
\usepackage{amsmath}
\usepackage{amssymb}
\begin{document}
\maketitle
\section{Aufgabe 84:}
Gegeben sind 3 verschiedene Farben 1,2,3. Wir färben den Graphen so, dass alle Knoten möglicherweise nur durch 2 Farben(also 1 und 2) gefärbt werden. Wenn es dringend die dritte Farbe benötigt wird, bedeutet es aber, dass dieser Knoten 2 Nachbarn mit verschiedenen Farben hat. Dann bestimmen wir die Anzahl der Knoten mit Farbe 1, Farbe 2 und Farbe 3 und nehmen das Maximum davon.\\

\textbf{Behauptung:}\\
G hat größte unabhängige Knotenmenge mit m Elementen, genau dann wenn es im Graphen G maximal m Knoten mit gleicher Farbe gibt.\\

\textbf{Beweis:}\\  
Wir haben m Knoten mit gleicher Farbe, angenommen Farbe 1. Nach der Färbung gemäß unserem Regel existiert es dann nur 2 folgende Fälle für die verbleibenden Knoten:
Dieser Knoten ist der Nachbar des einen von m Knoten. (In diesem Fall ist der Graph nur mit 2 Farben gefärbt)
Dieser Knoten ist der Nachbar von mehreren ($\geq$ 2)  Knoten. (Mit der dritten Farbe)
Also, es kann nicht mehr als m Knoten mit Farbe 1 geben $\rightarrow $ In diesem Graphen G kann nur m unabhängige Knoten geben  

\section{Aufgabe 87:}
$e^{-l_i}\geq1-l_i$ \textbf{(Nach der Definition)}\\
$\Leftrightarrow e^{-l_i}e^{L_i}e^{-L_i}\geq1-l_i$\\
$\Leftrightarrow e^{-l_i}e^{L_i}\geq(1-l_i)(1-L_i)$\\
$\Leftrightarrow e^{-l_i}\geq e^{-L_i}(1-l_i)(1-L_i)$\\
$\because$ $e^{-l_i}=2^{-\frac{l_i}{ln2}}$\\
$\Leftrightarrow \sqrt[ln2]{2^{-l_i}} \geq \sqrt[ln2]{2^{-L_i}}(1-l_i)(1-L_i)$\\
$\Leftrightarrow 2^{-l_i} \geq 2^{-L_i}(1-l_i)^{ln2}(1-L_i)^{ln2}$\\
\\
$(1-l_i)^{ln2}(1-L_i)^{ln2} \geq (1-ln2\cdot l_i)(1-{ln2}\cdot L_i) = 1 - ln2 \cdot l_i - ln2 \cdot L_i+ln^22\cdot L_il_i$\\ 
\textbf{(Nach der Definition von Binomischer Lehrsatz für natürliche Exponenten)}\\
Bleibt zu zeigen: \\
$-ln2\cdot L_i+ln^22 \cdot L_il_i \geq ln2 \cdot L_i$\\
$\Leftrightarrow L_i(ln2-1+l_i) \geq L_i$ \\
\textbf{Es ist sehr offensichtlich, dass $ln2-1+l_i$ größer als 1, weil die Tief immer größer als 1 ist.} \\
$\Rightarrow 2^{-l_i} \geq 2^{-L_1}(1-ln2 \cdot (l_i-L_i))$\\
$l_i := \log _2\left(\frac{1}{p_i}\right)$ einsetzen:\\
\\
$1=\sum 2^{-l_i} \geq \sum 2^{\log _2\left(\frac{1}{p_i}\right)}$\\
$\Rightarrow 1 \geq \sum p_i \cdot (1-ln2 \cdot l_i + ln2 \cdot \log _2\left(\frac{1}{p_i}\right))$\\
$\Leftrightarrow 1\geq \sum p_i - \sum p_i \cdot ln2\cdot l_i + \sum ln2 \cdot p_i \cdot \log _2\left(\frac{1}{p_i}\right)$\\
\textbf{Nach der Aufgabenstellung: }$\sum p_i = 1$\\
$\Rightarrow \sum ln2 \cdot p_i \cdot \log _2\left(\frac{1}{p_i}\right) \leq \sum p_i \cdot ln2\cdot l_i$\\
\textbf{Also, die mittlere Codewortlänge ist durch die Entropie von unten beschränkt}


\end{document}
